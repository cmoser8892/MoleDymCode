\chapter{Introduction}
\begin{comment}
- atoms maybe as the first bullet point 1800 know but also the greeks knew, but was kinda forgotten again. Do i need a cite there?! nah
- with stronger computers it became possible to simulate  the atoms, but kinda computationally expensive, methods have to be developed to simplifyc  stuff
- mention like the goal of this, build an own smol simulation
- introduction to md simulation 
- weird mix between chemistry, physics and computation + a lot of programming
- written in c++ instead of the standard python stuff 
- from a developers perspective programming in python will most likely take less time, while programming in be a bit faster (and more complicated)
\end{comment}

%This is the report for the class Molecular Dynamics with C++. The goal of the course was to build a small simulation witch is an introduction to the basis of molecular dynamics. 

Since around the early 1800 it is known that all matter that we can see is build from tiny particles, so called Atoms. With computers getting faster and faster, it is possible to simulate those atoms in a computer. But as with many things there has to be made trade off between accuracy of the simulation and being actually able to run said simulation in a reasonable time frame, while still getting correct results. 
Developing methods for this problem is the goal of molecular dynamics. 
\par 
In an introducing course to molecular dynamics the students were tasked to write their own simulation code in C++. The methods used, the implementation itself and results gained with the code are described in this report. 