\chapter{Conclusion}
\begin{comment}
- give look at stuff i did and did not
- mention all the other cool stuff, parralell hardware memes domain decompasition looking at other mechanical parameters ia stress
- should be written nice if i wanna work with those guys
-btw the github is actually hilarious although a c++ project over 80% is jupyther and other stuff lol 
- personal ?! nah
\end{comment}

Over the duration of the course we dove into a very interesting mix between chemistry, physics and programming, and we were also able to look at different methods for simulating liquids and solids.
\par 
First integration-schemes were used to be able to propagate the particles forward in space with a constant force affecting the atoms. In the next step potentials were introduced to calculated forces and energies affecting the atoms. Initially the Lenard-Jones-Potential was used to calculate forces and energies. Later the Embedded-Atoms-Method was used. With the introduction of Thermostats a stable configuration of the atoms could be found without the cluster melting or evaporating. 
\par
There would have been many more interesting themes to look further dive into, which could not have been looked into because of a lack of time. For example the program could have been modified via domain decomposition to better utilize modern parallel computer-architectures in a better way. 
