\chapter{Conclusion}
\begin{comment}
- give look at stuff i did and did not
- mention all the other cool stuff, parralell hardware memes domain decompasition looking at other mechanical parameters ia stress
- should be written nice if i wanna work with those guys
-btw the github is actually hilarious although a c++ project over 80% is jupyther and other stuff lol 
- personal ?! nah
\end{comment}

Over the duration of the course we dove into a very interesting mix between chemistry, physics and programming, and were able to look at different methods for simulating liquids and solids on an atomic scale.
\par 
First, integration-schemes were used to be able to propagate the particles forward in space with a constant force affecting the atoms. In the next step potentials were introduced to calculate forces and energies affecting the atoms. Initially the Lenard-Jones-Potential was used to calculate forces and energies. The potential was then expanded with a neighbor-hood list, to reduce the computation time. Later the Embedded-Atoms-Method was used. With the introduction of thermostats a stable configuration of atoms could be found without the risk of the cluster melting or evaporating. 
\par
Further improvements of the current implementation could include domain decomposition to utilize modern parallel computer-architectures and looking at other 
