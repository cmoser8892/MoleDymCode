\chapter{Implementation}
\begin{comment}	
	go about structure of the code 
	-> describe Code structure
		
	-> c++ was used to implement the code (mostly functional)
	-> key atoms container Class which holdes all the Values
	-> most functions were tested with googleTest
\end{comment}

%basic
\begin{comment}
code written in c++ most of it pretty functional, classes just used 
for the atoms container which holds the arrays 
while writing it also wrote the unittests with googletest
data aquiered form the code plotted with python
also where large simulations had to be run, called the program from the python code

--
code is structured into the milestones, so an individual milestone can be rerun in case of fuckup
parted into h and cpp files as usual

\end{comment}
%TODO cite shit
The simulation code was written in C++, most of it just as functions, although the positions, velocities, etc. of the individual atoms where saved in a container-class.
While writing the functions, these were also tested with unit-tests.
Plots generation and automation for running the project were written in python.

%used software
\begin{comment}
developed in CLion which as an integrated git inviroment
Clion builds with Cmake then clang as a compiler
debugger is gdb(nicely hidden)
- additianal bibs where :
	googletest	for unittests
	eigen		for arrays 
- 
\end{comment}
The C++-code was developed in CLion, an IDE which bundles many useful features together (CMake, GDB and Git).
The python-code was written in jupyter-notebook. 
Additional libaries used where: googletest for the unit-tests and eigen for the arrays used for data storage in the container-class.
