\chapter{Methods}
\begin{comment}
	0. Verlet Step
	1. Describe the Lenard Jones Potential
	2. Describe the Berendsen Thermostat
	3. Describe the Gupta 
	
\end{comment}
\begin{comment}
	Atoms are discreticed into the Postitions, Velocities and Forces 
	Second propagate the atom in the discriticed realm given constant Forces
		-> Velocity-Verlet Integration
	
	Compute Forces somehow
	-> Potentials
	LJ Potential 
	Gupta Potential
	
	Thermodynamic Effects
	Berendsen Thermostat
		->Gentle Way of resealing the Velocities
		
\end{comment}
%end general
\begin{comment}
goals of a md simulation
- determine the motion of each atom
	-> solve equation of motion 
	-> compute forces
	-> From the dirivation of the potential energy we can get the forces

- disciticed in time

\end{comment}
The goal of a molecular dynamics simulation is to determine the motion of each individual atom. For this we have to solve the equations of motion and compute the forces that affect each atom.
\begin{equation}
	\label{potentialForceEquation}
	\overrightarrow{f_{k}} = -\cfrac{\displaystyle\partial}{\displaystyle \partial \overrightarrow{r_{k}}} E_{pot}(\{\overrightarrow{r_{k}}\}) 
\end{equation}
The forces can be derived from the potential Energy as shown in \ref{potentialForceEquation} \cite[cf.][]{molDymCourse}. 
%TODO idont like
The simulation was run in a discretized space, so all updates to the motion of atoms was only done after a time step.


\section{Integration}
\begin{comment}
- obtain force from the previous equation
- dirivation in v. r.. from the mass
- system can be discribed just by the velocity and postion of each atom
- in the simulation the positions and velocities of the individual atoms have to be tracked and updated accordingly
- most used  is the Velocity-Verlet Algorithm
\end{comment}
From the equation \ref{potentialForceEquation} we can obtain the force at each individual atom with it's acceleration, velocity and position as shown in  \ref{potentialForceEquation}. 
\begin{equation}
	\label{potentialForceEquationForEachAtom}
	\overrightarrow{f_{i}} = \cfrac{\displaystyle\partial E_{pot}}{\displaystyle \partial \overrightarrow{r_{i}}} = m_{i}\overrightarrow{a_{i}} = m_{i}\overrightarrow{\dot{v_{i}}} = m_{i}\overrightarrow{\ddot{r_{i}}}
\end{equation}


\section{Potentials}
\begin{comment}
- from the potential energy the forces affection each atom can be derived
- the forces which affect the atoms than determine the motion of each atom
\end{comment}

%General Formula for the dirivation
\begin{equation}
	\overrightarrow{f_{k}} = -\frac{\partial E_{pot}}{\partial  \overrightarrow{r_{k}}}=\sum_{i}^{}\frac{\partial V}{\partial r_{ik}} \hat{r_{ik}}
\end{equation}


\subsection{Lenard-Jones-Potential}
\begin{comment}
- pair potential
\end{comment}
%Formulation of the lj potential
\begin{equation}
	V(r) = 4\epsilon\bigg[\Big(\frac{\sigma}{r}\Big)^{12}- \Big(\frac{\sigma}{r}\Big)^{6} \bigg]
\end{equation}
%Potential derivation with python
\begin{tcolorbox}[breakable, size=fbox, boxrule=1pt, pad at break*=1mm,colback=cellbackground, colframe=cellborder]
\prompt{In}{incolor}{4}{\boxspacing}
\begin{Verbatim}[commandchars=\\\{\}]
\PY{k+kn}{import} \PY{n+nn}{sympy} \PY{k}{as} \PY{n+nn}{sp}
\PY{k+kn}{import} \PY{n+nn}{warnings}
\PY{n}{warnings}\PY{o}{.}\PY{n}{filterwarnings}\PY{p}{(}\PY{l+s+s1}{\PYZsq{}}\PY{l+s+s1}{ignore}\PY{l+s+s1}{\PYZsq{}}\PY{p}{)}
\PY{n}{sp}\PY{o}{.}\PY{n}{init\PYZus{}printing}\PY{p}{(}\PY{p}{)}
\PY{n}{eps} \PY{o}{=} \PY{n}{sp}\PY{o}{.}\PY{n}{Symbol}\PY{p}{(}\PY{l+s+s2}{\PYZdq{}}\PY{l+s+s2}{e}\PY{l+s+s2}{\PYZdq{}}\PY{p}{)}
\PY{n}{sig} \PY{o}{=} \PY{n}{sp}\PY{o}{.}\PY{n}{Symbol}\PY{p}{(}\PY{l+s+s2}{\PYZdq{}}\PY{l+s+s2}{s}\PY{l+s+s2}{\PYZdq{}}\PY{p}{)}
\PY{n}{rad} \PY{o}{=} \PY{n}{sp}\PY{o}{.}\PY{n}{Symbol}\PY{p}{(}\PY{l+s+s2}{\PYZdq{}}\PY{l+s+s2}{r}\PY{l+s+s2}{\PYZdq{}}\PY{p}{)}
\PY{n}{energyRad} \PY{o}{=} \PY{l+m+mi}{4} \PY{o}{*} \PY{n}{eps} \PY{o}{*} \PY{p}{(}\PY{p}{(}\PY{n}{sig}\PY{o}{/}\PY{n}{rad}\PY{p}{)}\PY{o}{*}\PY{o}{*}\PY{l+m+mi}{12} \PY{o}{\PYZhy{}} \PY{p}{(}\PY{n}{sig}\PY{o}{/}\PY{n}{rad}\PY{p}{)}\PY{o}{*}\PY{o}{*}\PY{l+m+mi}{6}\PY{p}{)}
\PY{n}{energyRad}\PY{o}{.}\PY{n}{diff}\PY{p}{(}\PY{n}{rad}\PY{p}{)}
	\end{Verbatim}
\end{tcolorbox}
\prompt{Out}{outcolor}{4}{}

$\displaystyle 4 e \left(\frac{6 s^{6}}{r^{7}} - \frac{12 s^{12}}{r^{13}}\right)$

\subsection{Neighborhood-Search Algorithm}
%Not sure about that one
\begin{comment}

\end{comment}

\subsection{Embedded-Atom Method Potentials}
\begin{comment}
- describe Units 
- will be funny describing it in general
\end{comment}

\subsection{Berendsen-Thermostat}

\begin{equation}
	\lambda = \sqrt{1 + \bigg(\frac{T_{0}}{T} -1\bigg)\frac{\Delta t}{\tau}}
\end{equation}

\begin{equation}
	T(t) = T_{0} + (T_{1}-T_{0})e^{-t/\tau}
\end{equation}

