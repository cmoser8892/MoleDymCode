\chapter{Methods}
\begin{comment}
	0. Verlet Step
	1. Describe the Lenard Jones Potential
	2. Describe the Berendsen Thermostat
	3. Describe the Gupta 
	
\end{comment}
\begin{comment}
	Atoms are discreticed into the Postitions, Velocities and Forces 
	Second propagate the atom in the discriticed realm given constant Forces
		-> Velocity-Verlet Integration
	
	Compute Forces somehow
	-> Potentials
	LJ Potential 
	Gupta Potential
	
	Thermodynamic Effects
	Berendsen Thermostat
		->Gentle Way of resealing the Velocities
		
\end{comment}
%end general
\begin{comment}
goals of a md simulation
- determine the motion of each atom
	-> solve equation of motion 
	-> compute forces
	-> From the dirivation of the potential energy we can get the forces

- disciticed in time

\end{comment}
The goal of a molecular dynamics simulation is to determine the motion of each individual atom. To achieve this we have to solve the equations of motion and compute the forces that affect each atom.
\begin{equation}
	\label{potentialForceEquation}
	\overrightarrow{f_{k}}(\overrightarrow{r_{k}}) = -\cfrac{\displaystyle\partial}{\displaystyle \partial \overrightarrow{r_{k}}} E_{\mathrm{pot}}(\{\overrightarrow{r_{k}}\}) 
\end{equation}
The forces can be derived from the potential energy as shown in Eq.  \ref{potentialForceEquation} \cite[][]{molDymCourse}. 


\section{Integration}
\begin{comment}
- obtain force from the previous equation
- dirivation in v. r.. from the mass
- system can be discribed just by the velocity and postion of each atom
- in the simulation the positions and velocities of the individual atoms have to be tracked and updated accordingly
- most used  is the Velocity-Verlet Algorithm
\end{comment}
From Eq. \ref{potentialForceEquation} we can obtain the force acting on each individual atom with its acceleration, velocity and position as shown in  Eq. \ref{potentialForceEquationForEachAtom}. 
\begin{equation}
	\label{potentialForceEquationForEachAtom}
	\overrightarrow{f_{i}} = \cfrac{\displaystyle\partial E_{\mathrm{pot}}}{\displaystyle \partial \overrightarrow{r_{i}}} = m_{i}\overrightarrow{a_{i}} = m_{i}\overrightarrow{\dot{v_{i}}} = m_{i}\overrightarrow{\ddot{r_{i}}}
\end{equation}
As is shown the system can be described just by the velocities and positions of the atoms. 
In the implementation we hold these values in a container-class to ensure that they are consecutive in  memory, which speeds up the simulation. 
\par
%to porpagate
In the next step we need to propagate the simulation forward in time. A popular algorithms to update the positions and velocities is the Velocity-Verlet integration algorithm.
The scheme is often split up into two steps, the prediction step shown in the Eqs.  \ref{verletPrediction1} and \ref{verletPrediction2}, and the correction step shown in Eq. \ref{verletCorrection} \cite[cf. ][]{molDymCourse}. The force which is needed in the second step can be updated beforehand, based on the new positions from the first step.
%Verlet steps
\begin{equation}
	\label{verletPrediction1}
	\overrightarrow{v_{i}}(t+\Delta t/2) = 
	\overrightarrow{v_{i}}(t) + 
	\frac{\overrightarrow{f_{i}}(t)\Delta t}{2m_{i}}
\end{equation}

\begin{equation}
	\label{verletPrediction2}
	\overrightarrow{r_{i}}(t+ \Delta t) = 
	\overrightarrow{r_{i}}(t) + \overrightarrow{v_{i}}(t + \Delta t/2)\Delta t
\end{equation}

\begin{equation}
	\label{verletCorrection}
	\overrightarrow{v_{i}}(t+\Delta t) = \overrightarrow{v_{i}}(t+\Delta t/2) +
	\frac{\overrightarrow{f_{i}}(t + \Delta t)\Delta t}{2m_{i}}
\end{equation}
We are now able to propagate atoms forward with a constant force, therefore we have to actually model the forces affecting each atom next.


\section{Lenard-Jones-Potential}
\begin{comment}
- pair potential
\end{comment}
%General Formula for the dirivation
A popular pair potential to model interatomic interactions is the Lenard-Jones-Potential. 
The goal is to model the Coulomb force and the Pauli repulsion. 
\begin{equation}
	V(r) = 4\epsilon\bigg[\Big(\frac{\sigma}{r}\Big)^{12}- \Big(\frac{\sigma}{r}\Big)^{6} \bigg]
\end{equation}
As shown in Eq. \ref{potentialForceEquation} we can derive the forces from the potential energy. 
In Eq. \ref {potentialEquationAtoms} we formulate the equation for each atom. 
We can acquire the force affecting each atom if we sum up the derivative of the potential energy between atom k and all the other atoms i times the normed vector to atom i from atom k. 
\begin{equation}
	\label{potentialEquationAtoms}
	\overrightarrow{f_{k}} = -\frac{\partial E_{\mathrm{pot}}}{\partial  \overrightarrow{r_{k}}} = \sum_{i}^{}\frac{\partial V}{\partial r_{ik}} \hat{r_{ik}}
\end{equation}
%Formulation of the lj potential
To implement the equation we have to derive $\delta V/ \delta r_{ik}$ analytically, which was done using the symbolic python library sympy. 
%Potential derivation with python
\begin{tcolorbox}[breakable, size=fbox, boxrule=1pt, pad at break*=1mm,colback=cellbackground, colframe=cellborder]
%\prompt{In}{incolor}{4}{\boxspacing}
\begin{Verbatim}[commandchars=\\\{\}]
\PY{k+kn}{import} \PY{n+nn}{sympy} \PY{k}{as} \PY{n+nn}{sp}
\PY{k+kn}{import} \PY{n+nn}{warnings}
\PY{n}{warnings}\PY{o}{.}\PY{n}{filterwarnings}\PY{p}{(}\PY{l+s+s1}{\PYZsq{}}\PY{l+s+s1}{ignore}\PY{l+s+s1}{\PYZsq{}}\PY{p}{)}
\PY{n}{sp}\PY{o}{.}\PY{n}{init\PYZus{}printing}\PY{p}{(}\PY{p}{)}
\PY{n}{eps} \PY{o}{=} \PY{n}{sp}\PY{o}{.}\PY{n}{Symbol}\PY{p}{(}\PY{l+s+s2}{\PYZdq{}}\PY{l+s+s2}{e}\PY{l+s+s2}{\PYZdq{}}\PY{p}{)}
\PY{n}{sig} \PY{o}{=} \PY{n}{sp}\PY{o}{.}\PY{n}{Symbol}\PY{p}{(}\PY{l+s+s2}{\PYZdq{}}\PY{l+s+s2}{s}\PY{l+s+s2}{\PYZdq{}}\PY{p}{)}
\PY{n}{rad} \PY{o}{=} \PY{n}{sp}\PY{o}{.}\PY{n}{Symbol}\PY{p}{(}\PY{l+s+s2}{\PYZdq{}}\PY{l+s+s2}{r}\PY{l+s+s2}{\PYZdq{}}\PY{p}{)}
\PY{n}{energyRad} \PY{o}{=} \PY{l+m+mi}{4} \PY{o}{*} \PY{n}{eps} \PY{o}{*} \PY{p}{(}\PY{p}{(}\PY{n}{sig}\PY{o}{/}\PY{n}{rad}\PY{p}{)}\PY{o}{*}\PY{o}{*}\PY{l+m+mi}{12} \PY{o}{\PYZhy{}} \PY{p}{(}\PY{n}{sig}\PY{o}{/}\PY{n}{rad}\PY{p}{)}\PY{o}{*}\PY{o}{*}\PY{l+m+mi}{6}\PY{p}{)}
\PY{n}{energyRad}\PY{o}{.}\PY{n}{diff}\PY{p}{(}\PY{n}{rad}\PY{p}{)}
	\end{Verbatim}
\end{tcolorbox}

With the codesnippet from above the derivative of the potential was obtained as shown in Eq. \ref{potentialEquationAtomsDirivative}.
\begin{equation}
	\label{potentialEquationAtomsDirivative}
	\frac{\partial V}{\partial r} = 4 \epsilon \left(\frac{6 \sigma^{6}}{r^{7}} - \frac{12 \sigma^{12}}{r^{13}}\right)
\end{equation}
The result from Eq. \ref{potentialEquationAtomsDirivative} can be used to calculate the interatomic forces from Eq. \ref {potentialEquationAtoms}. 
We can optimize the implementation by subtracting the force that affects atom k from the forces of atom i, as they have to be the same just with the opposite direction and therefore sign. 
\par 
It can be seen that in the simulation this algorithm behaves in the order O(N²) with increasing numbers of atoms N. It can be reduced to a linear order O(N) if we only consider atoms up to a certain distance (cutoff-distance) around each atom. This a good approximation as the forces get smaller as the interatomic distance between the atoms increases. The next goal is to generate a list containing the nearest atoms that are inside this cutoff-distance, which can be done by using domain decomposition \cite[cf.][]{molDymCourse}
\section{Berendsen-Thermostat}
\begin{comment}
- couple the moleclular system to a larger heat bath
- thermostat controls the heat of the simulation so the system does not melt or evaporate
\end{comment}
The initial state of the atomic system is often far away form the equilibrium. To prevent atoms from melting or evaporating, thermostats are used to control the heat/energy of the atoms. A rather simple thermostat, which was used in the simulation, is the Berendsen-thermostat. 
\par
It is typically implemented by rescaling the velocities by a certain factor as shown in Eq.  \ref{coupelingFunction}. The factor is composed of the target-temperature $T_{0}$, the current temperature $T$, the timestep of the simulation $\Delta{t}$ and the relaxation time $\tau$. 
\begin{equation}
	\label{coupelingFunction}
	\lambda = \sqrt{1 + \bigg(\frac{T_{0}}{T} -1\bigg)\frac{\Delta t}{\tau}}
\end{equation}
Reducing the temperature of the simulation via rescaling the velocities of each atom can be done because the temperature is connected to kinetic energy via the following equation \cite[cf.][]{molDymCourse}. 
\begin{equation}
	\label{bolzmann}
	\frac{3}{2} k_{\mathrm{B}} T = \sum_{i} \frac{1}{2} m v_{i}^2
\end{equation}
With Eq. \ref{bolzmann} the current temperature can be calculated and put into Eq. \ref{coupelingFunction}.


\section{Embedded-Atom Method Potentials}
\begin{comment}
- describe Units 
- better discribed in the course
- good model for metalic systems
- work with gold clusters 

\end{comment}
Embedded atom-potentials are a good model for metallic bonding in solid or liquid states for different materials. The method used in the code was developed by Gupta \cite{gupta} and the parameters were taken from  Cleri \& Rosato \cite{rosato}. 
\par 
%TODO explain a bit
\begin{comment}
pair potentials cant model crystals as they have 3 independet elastic constants
put pair potentials have only two -> cauchy pressure
rather heuristic description in the lecture, like basic chemistry
E_pot = E_repulsion + E_embedding (embedding energy is connected to electric density)
E_repulustion is typically a pair potential
E_emedding is a funtiunal of the density in r
\end{comment}
Compared to pair potentials, the potential energy is calculated from two contributions shown in Eq. \ref{EAM-basic}. The first contribution $E_{repulsion}$ is modeled as a pair-potential. The second factor $E_{bonding}$ models the electron density between the metal atoms. 
\begin{equation}
	\label{EAM-basic}
	E_{\mathrm{pot}}  = E_{\mathrm{repulsion}} + E_{\mathrm{bonding}}
\end{equation}
 
The Eqs. \ref{rosatto-repulsion} and \ref{rosatto-bonding} show how $E_{\mathrm{repulsion}}$ and $E_{\mathrm{bonding}}$ were implemented in the code. The parameter $A$,  $\xi$, $p$ and $q$ are also taken from the paper of Cleri and Rosato and can be fitted to different types of metals \cite{rosato}. 
\begin{equation}
 	\label{rosatto-repulsion}
 	\mathrm{E}_{\mathrm{R}}^{i} = \sum_{j} A e^{-p(r_ij/r_{0} - 1)}
 \end{equation}

\begin{equation}
	\label{rosatto-bonding}
	%\mathrm{}_{}^{}
	\mathrm{E}_{\mathrm{B}}^{i} = - \big\{\sum_{j}\xi^{2} e^{-2q(r_{ij}/r_{0}-1)} \big\}^{1/2}
\end{equation}

To finalize the implementation, the forces between the atoms also have to be calculated. For this the Eq. \ref{potentialEquationAtoms} can be considered again. The expression has to be derived.
